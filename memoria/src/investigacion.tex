\section{Caso de investigación: identificación de partes de un documento.}
Una aplicación de FOIL puede ser la generación de reglas para el reconocimiento de partes de un documento. La aplicación de esta idea a cartas formales es un clásico, pero muy fácilmente se le podría dar un enfoque más actual usandolo para identificar partes de un documeto estructurado como html, xml, json, etc.

Los datos que podemos encontrar en repositorios online definen varios conceptos, expresados como predicados a aprender. Se tratan de pártes lógicas que se pueden identificar en los documentos, como por ejemplo en nombre de quien lo envía, el destinatario, el logotipo, la fecha, etc.

Las relaciones usadas para describir la información son del tipo:

\begin{table}
  \setlength{\tabcolsep}{15pt} % Default value: 6pt
  \begin{tabular}{l l l}
    ancho-muy-muy-pequeño(x) & ancho-muy-pequeño(x) & ancho-pequeño(x) \\
    ancho-medio-pequeño(x) & ancho-medio(x) & ancho-medio-grande(x) \\
    ancho-grande(x) & ancho-muy-grande(x) & ancho-muy-muy-grande(x) \\
  \end{tabular}
\end{table}
para referirse a el ancho de un bloque x, usando predicados similares para el alto, o relaciones del tipo:

\begin{table}
  \setlength{\tabcolsep}{20pt} % Default value: 6pt
  \begin{tabular}{l l l}
    tipo-texto(x) & tipo-linea-horiz(x) & tipo-line-vert(x) \\
    tipo-imagen(x) & tipo-grafico(x) & tipo-mezcla(x) \\
  \end{tabular}
\end{table}
para describir el tipo del bloque x.

Esto se adapta muy bien a los tipos de documentos que se usan en la web, puesto que son documentos estucturados, digitales, pero sin embargo la inmensa mayoría de ellos no tienen semántica asociada a la estuctura.

Documentos que usan el estandar html5 empiezan a incluir etiquetas semánticas, pero aún así, la gran mayoria de la información en la web no las usa aún. Conseguir datos para experimentar se tremendamente sencillo.

Otro enfoque que se le podría dar es la identificación de las estucturas optimas para interfaces web, mediante un conjunto de datos que incluya diferentes parámetros medidos a partir del uso de los usuarios, como por ejemplo los clics en determinados enlaces, las visitas a determinada pagina, etc.
