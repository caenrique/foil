\section{Introducción}
La programación lógica inductiva se basa en la lógica de primer orden para desarrollar teorías. Los datos de entrenamiento constan de una relación objetivo, definida extensionalmente con un nombre y una tupla de terminos constantes, y un conjunto de infomación de fondo definido a su vez como relaciones. En el Desarrollo original de FOIL se posibilitaba que esta información de fondo estuviera definida tanto extensionalmente como intencionalmente, es decir, por relaciones contantes cubriendo todos los casos positivos de esa relación, o mediante otras relaciones, respectivamente. En este trabajo solo se permite la definición de dichas relaciones extensionalmente. Al igual que en otros sitemas basados en arboles de decisión, y pares atributo-valor, los ejemplos que pertenecen y los que no pertenecen a la relación objetivo los denominamos $\oplus$ y $\ominus$ respectivamente.


Decimos que una teoría completa cubre a una tupla si la evaluación de al menos una de sus reglas es satisfactoria. El objetivo, por tanto, de los sistemas de apredizade basados en lógica de primer orden es construir teorías que cubren a todos los ejemplos $\oplus$ y excluyen a todos los ejemplos $\ominus$.


FOIL es un sistema de aprendizaje lógico inductivo que presenta una estrategia \emph{separate-and-conquer}. Así mismo, el método usado para encontrar una regla adecuada es \emph{top-down}, es decir, partiendo de una regla sin cuerpo (que cubre a todos los ejemplos), va añadiendo cláusulas que van restringiendo la cobertura hasta que no cubre ningún ejemplo $\ominus$.
